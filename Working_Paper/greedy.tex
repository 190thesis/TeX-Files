\subsection{Greedy Algorithm}
\begin{algorithm}
			\caption{Greedy Algorithm}
			\begin{algorithmic}[1]
				\Require $G=(V,E),$ Thresholds $ t(v)$ for $v\in V$
				\Ensure Target Set $S$
				\State $S=\varnothing$
				\State 
				$U=V$
				\ForAll{$v \in v$}
					\State $\delta(v)=d(v)$
					\State $k(v)=t(v)$
					\State	$N(v)=\Gamma(v)$
				\EndFor
				\While{$U \not = \varnothing$}
					\State $v=argmax_{u\in U}\{\delta(u)\}$
					\If{$k(v)>0$}
						\State $v=argmax_{u\in U}\{k(u)\}$
						\State $S=S\cup\{v\}$
					\EndIf
					\ForAll{$u \in N(v)$}
						\State $\delta(u)=\delta(u)-1$
						\State $N(u)=N(u)-{v}$
					\EndFor
					\State $U=U-{v}$
				\EndWhile
			\end{algorithmic}
		\end{algorithm}
		
		The Greedy version of the TSS algorithm picks the most likely vertex to be in the target set, thus the name. The algorithm finds the vertex with the highest threshold but includes the node highest degree instead. This ensures that if the thresholds of the nodes are the same but the degrees differ, we still want to to pick the most likely node to be in the target set. The algorithm runs slow at the start but picks up its speed since we consider the nodes with the highest degrees first, meaning it has a high neighbor count needing a lot of time to update said neighbors properties.
\newpage

\subsubsection{Greedy Step by step example}
\begin{figure}[h!]
	\centering
	\begin{tikzpicture}[-,>=stealth',shorten >=1pt,auto,node distance=3cm,
                    thick,main node/.style={circle,draw,font=\sffamily\small\bfseries}]

  \node[main node,label={right:$v_1$}] (1) {$3$};
  \node[main node,label={right:$v_2$},below of=1] (2) {$1$};
  \node[main node,label={right:$v_3$}, below right of=1] (3) {$1$};
  \node[main node,label={right:$v_4$},above right of=1] (4) {$1$};
  \node[main node,label={right:$v_5$}, below right of=4] (5) {$2$};
  \node[main node,label={right:$v_6$},below of=5] (6) {$1$};
  \node[main node,label={right:$v_7$},below left of=1] (7) {$2$};
  \node[main node,label={right:$v_8$},below left of=7] (8) {$1$};
  \node[main node,label={right:$v_9$},below right of=7] (9) {$1$};
  \node[main node,label={right:$v_{10}$},below of=6] (10) {$1$};

  \path[every node/.style={font=\sffamily\large}]
    (1) edge node {} (7)
    	edge node {} (2)
    	edge node {} (3)
    (2) 
    (3) 
    (4) edge node {} (1)
    	edge node {} (5)
    (5) edge node {} (6)
    (6) edge node {} (10)
    (7) edge node {} (8)
    	edge node {} (9)
    (8)
    (9)
    (10) 
    
    ;
\end{tikzpicture}
\caption{A tree with 10 vertices where the number inside the circle is the vertex threshold ($t(v)$).}

\centering
    \begin{tabular}{c c c c}
		Vertex $v_{i}$ & Degree $d^{in}_{i}$ & Threshold $t_{i}$ \\
		$v_{1}$ & 4 & 3  \\
		$v_{2}$ & 1 & 1  \\
		$v_{3}$ & 1 & 1  \\
		$v_{4}$ & 2 & 1   \\
		$v_{5}$ & 2 & 2   \\
		$v_{6}$ & 2 & 1  \\
		$v_{7}$ & 3 & 2  \\
		$v_{8}$ & 1 & 1 \\
		$v_{9}$ &  1 & 1 \\
		$v_{10}$ & 1 & 1 \\
	\end{tabular}
\end{figure}

 \begin{minipage}{0.45\textwidth}
 	\centering
	\begin{tikzpicture}[-,>=stealth',shorten >=1pt,auto,node distance=1.7cm,
                    thick,main node/.style={circle,draw,font=\sffamily\small\bfseries}]

  \node[main node,label={right:$v_1$},fill=orange] (1) {$0$};
  \node[main node,label={right:$v_2$},below of=1,fill=yellow] (2) {$0$};
  \node[main node,label={right:$v_3$}, below right of=1,fill=yellow] (3) {$0$};
  \node[main node,label={right:$v_4$},above right of=1,fill=yellow] (4) {$0$};
  \node[main node,label={right:$v_5$}, below right of=4,fill=yellow] (5) {$1$};
  \node[main node,label={right:$v_6$},below of=5] (6) {$1$};
  \node[main node,label={right:$v_7$},below left of=1,fill=yellow] (7) {$1$};
  \node[main node,label={right:$v_8$},below left of=7] (8) {$1$};
  \node[main node,label={right:$v_9$},below right of=7] (9) {$1$};
  \node[main node,label={right:$v_{10}$},below of=6] (10) {$1$};
  
  \path[every node/.style={font=\sffamily\large}]
    (1) edge node {} (7)
    	edge node {} (2)
    	edge node {} (3)
    (2) 
    (3) 
    (4) edge node {} (1)
    	edge node {} (5)
    (5) edge node {} (6)
    (6) edge node {} (10)
    (7) edge node {} (8)
    	edge node {} (9)
    (8)
    (9)
    (10) 
    ;
\end{tikzpicture}
\end{minipage}
\begin{minipage}{0.55\textwidth}
	\textbf{Iteration 1 ($v_1$)}
	
    \begin{tabular}{c c c c}
		Vertex $v_{i}$ & Degree $d^{in}_{i}$ & Threshold $t_{i}$\\
		$v_{1}$ & 0 & 0 \\
		$v_{2}$ & 0 & 0 \\
		$v_{3}$ & 0 & 0 \\
		$v_{4}$ & 0 & 0 \\
		$v_{5}$ & 1 & 1 \\
		$v_{6}$ & 2 & 1 \\
		$v_{7}$ & 2 & 1 \\
		$v_{8}$ & 1 & 1  \\
		$v_{9}$ &  1 & 1 \\
		$v_{10}$ & 1 & 1\\
	\end{tabular}

\end{minipage}
\noindent\makebox[\linewidth]{\rule{\paperwidth}{0.4pt}}
\vspace*{0.3cm}  
  \begin{minipage}{0.45\textwidth}
 	\centering
	\begin{tikzpicture}[-,>=stealth',shorten >=1pt,auto,node distance=1.7cm,
                    thick,main node/.style={circle,draw,font=\sffamily\small\bfseries}]

  \node[main node,label={right:$v_1$}] (1) {$0$};
  \node[main node,label={right:$v_2$},below of=1] (2) {$0$};
  \node[main node,label={right:$v_3$}, below right of=1] (3) {$0$};
  \node[main node,label={right:$v_4$},above right of=1] (4) {$0$};
  \node[main node,label={right:$v_5$}, below right of=4,fill=yellow] (5) {$0$};
  \node[main node,label={right:$v_6$},below of=5,fill=orange] (6) {$0$};
  \node[main node,label={right:$v_7$},below left of=1] (7) {$1$};
  \node[main node,label={right:$v_8$},below left of=7] (8) {$1$};
  \node[main node,label={right:$v_9$},below right of=7] (9) {$1$};
  \node[main node,label={right:$v_{10}$},below of=6,fill=yellow] (10) {$0$};
  
  \path[every node/.style={font=\sffamily\large}]
    (1) edge node {} (7)
    	edge node {} (2)
    	edge node {} (3)
    (2) 
    (3) 
    (4) edge node {} (1)
    	edge node {} (5)
    (5) edge node {} (6)
    (6) edge node {} (10)
    (7) edge node {} (8)
    	edge node {} (9)
    (8)
    (9)
    (10) 
    ;
\end{tikzpicture}
\end{minipage}
\begin{minipage}{0.55\textwidth}
	\textbf{Iteration 2 ($v_6$)}
	
    \begin{tabular}{c c c c}
		Vertex $v_{i}$ & Degree $d^{in}_{i}$ & Threshold $t_{i}$\\
		$v_{1}$ & 0 & 0 \\
		$v_{2}$ & 0 & 0 \\
		$v_{3}$ & 0 & 0 \\
		$v_{4}$ & 0 & 0 \\
		$v_{5}$ & 0 & 0 \\
		$v_{6}$ & 0 & 0 \\
		$v_{7}$ & 1 & 1 \\
		$v_{8}$ & 1 & 1  \\
		$v_{9}$ &  1 & 1 \\
		$v_{10}$ & 0 & 0\\
	\end{tabular}

\end{minipage}
\noindent\makebox[\linewidth]{\rule{\paperwidth}{0.4pt}}
\vspace*{0.3cm}  

  \begin{minipage}{0.45\textwidth}
 	\centering
	\begin{tikzpicture}[-,>=stealth',shorten >=1pt,auto,node distance=1.7cm,
                    thick,main node/.style={circle,draw,font=\sffamily\small\bfseries}]

  \node[main node,label={right:$v_1$}] (1) {$0$};
  \node[main node,label={right:$v_2$},below of=1] (2) {$0$};
  \node[main node,label={right:$v_3$}, below right of=1] (3) {$0$};
  \node[main node,label={right:$v_4$},above right of=1] (4) {$0$};
  \node[main node,label={right:$v_5$}, below right of=4] (5) {$0$};
  \node[main node,label={right:$v_6$},below of=5] (6) {$0$};
  \node[main node,label={right:$v_7$},below left of=1,fill=orange] (7) {$0$};
  \node[main node,label={right:$v_8$},below left of=7,fill=yellow] (8) {$0$};
  \node[main node,label={right:$v_9$},below right of=7,fill=yellow] (9) {$0$};
  \node[main node,label={right:$v_{10}$},below of=6] (10) {$0$};
  
  \path[every node/.style={font=\sffamily\large}]
    (1) edge node {} (7)
    	edge node {} (2)
    	edge node {} (3)
    (2) 
    (3) 
    (4) edge node {} (1)
    	edge node {} (5)
    (5) edge node {} (6)
    (6) edge node {} (10)
    (7) edge node {} (8)
    	edge node {} (9)
    (8)
    (9)
    (10) 
    ;
\end{tikzpicture}
\end{minipage}
\begin{minipage}{0.55\textwidth}
	\textbf{Iteration 3 ($v_7$)}
	
    \begin{tabular}{c c c c}
		Vertex $v_{i}$ & Degree $d^{in}_{i}$ & Threshold $t_{i}$\\
		$v_{1}$ & 0 & 0 \\
		$v_{2}$ & 0 & 0 \\
		$v_{3}$ & 0 & 0 \\
		$v_{4}$ & 0 & 0 \\
		$v_{5}$ & 0 & 0 \\
		$v_{6}$ & 0 & 0 \\
		$v_{7}$ & 0 & 0 \\
		$v_{8}$ & 0 & 0  \\
		$v_{9}$ &  0 & 0 \\
		$v_{10}$ & 0 & 0\\
	\end{tabular}

\end{minipage}
\noindent\makebox[\linewidth]{\rule{\paperwidth}{0.4pt}}
\vspace*{0.3cm}  

We can stop here since all of the degrees of the vertices in the graph are 0. We get $S=\{v_{1},v_{6},v_{7}\}$. 